\documentclass[12pt]{article}

\usepackage{fullpage}
\usepackage{multicol,multirow}
\usepackage{tabularx}
\usepackage{ulem}
\usepackage[utf8]{inputenc}
\usepackage[russian]{babel}
\usepackage{amsmath}
\usepackage{amsmath}

\usepackage{titlesec}

\titleformat{\section}
  {\normalfont\Large\bfseries}{\thesection.}{0.3em}{}

\titleformat{\subsection}
  {\normalfont\large\bfseries}{\thesubsection.}{0.3em}{}

\titlespacing{\section}{0pt}{*2}{*2}
\titlespacing{\subsection}{0pt}{*1}{*1}
\titlespacing{\subsubsection}{0pt}{*0}{*0}
\usepackage{listings}
\lstloadlanguages{Lisp}
\lstset{extendedchars=false,
	breaklines=true,
	breakatwhitespace=true,
	keepspaces = true,
	tabsize=2
}
\begin{document}


\section*{Отчет по лабораторной работе №\,1
по курсу \guillemotleft  Функциональное программирование\guillemotright}
\begin{flushright}
Студент группы 8О-308 МАИ \textit{Марков Александр}, \textnumero 15 по списку \\
\makebox[7cm]{Контакты: {\tt markov.lifeacc@gmail.com} \hfill} \\
\makebox[7cm]{Работа выполнена: 11.03.2021 \hfill} \\
\ \\
Преподаватель: Иванов Дмитрий Анатольевич, доц. каф. 806 \\
\makebox[7cm]{Отчет сдан: \hfill} \\
\makebox[7cm]{Итоговая оценка: \hfill} \\
\makebox[7cm]{Подпись преподавателя: \hfill} \\

\end{flushright}

\section{Тема работы}
Примитивные функции и особые операторы Common Lisp.

\section{Цель работы}
Научиться вводить S-выражения в Лисп-систему, определять переменные и функции, работать с условными операторами, работать с числами, использую схему линейной и древовидной рекурсии.

\section{Задание (вариант №1.27)}
Функция f определяется правилом:

\begin{equation*}
f(n) =
 \begin{cases}
    n&\text{, если n < 3}\\
    f(n-1) + f(n-2) + f(n-3)&\text{, если n $\geq$ 3}
 \end{cases}
\end{equation*}

Запрограммируйте на языке Коммон Лисп функцию, вычисляющую f с помощью линейно-рекурсивного процесса. Оцените требуемые время вычисления и оперативную память.

\section{Оборудование студента}
Процессор AMD Ryzen 5 4600H 3.00 GHz, память: 16Gb, разрядность системы: 64.

\section{Программное обеспечение}
ОС Windows 10, среда LispWorks Personal Edition 7.1.2

\section{Идея, метод, алгоритм}
Для вычисления функции f используется вспомогательная линейно-рекурсивная функция f\_help, которая имеет следующие параметры:

На каждом i-ом шаге рекурсии
\begin{itemize}
    \setlength{\itemsep}{-1mm}
    \item p1 - число, равное f(i);
    \item p2 - число, равное f(i + 1);
    \item p3 - число, равное f(i + 2);
    \item n - количество оставшихся шагов рекурсии
\end{itemize}

Функция f\_help работает следующим образом:
\begin{itemize}
    \item Если n = 0, то найдено искомое число, оно находится в переменной p1.
    \item Если n > 0, то находим f(i + 3) и сохраняем значение f(i + 1) в p1, f(i + 2) в p2 и f(i + 3) в p3 и вызываем функцию f\_help(p1, p2, p3, n - 1).
\end{itemize}

Таким образом, чтобы найти значение f(n), мы вызываем функцию f\_help(0, 1, 2, n), после n рекурсивных вызовов f\_help, в p1 будет находиться значение f(n).

\section{Сценарий выполнения работы}

\section{Распечатка программы и её результаты}

\subsection{Исходный код}
\begin{lstlisting}
(defun f_help (p1 p2 p3 n)
  (if (= n 0)
      p1
      (f_help p2 p3 (+ p1 p2 p3) (- n 1))
  )
)

(defun f (n)
  (f_help 0 1 2 n)
)
\end{lstlisting}

\subsection{Результаты работы}

\begin{lstlisting}
CL-USER 1 > (f 0)
0
CL-USER 2 > (f 1)
1
CL-USER 3 > (f 2)
2
CL-USER 4 > (f 3)
3
CL-USER 5 > (f 4)
6
CL-USER 6 > (f 5)
11
CL-USER 7 > (f 6)
20
CL-USER 8 > (f 7)
37
CL-USER 9 > (f 8)
68
CL-USER 10 > (f 9)
125
CL-USER 11 > (f 10)
230
\end{lstlisting}

\section{Дневник отладки}
\begin{tabular}{|c|c|c|c|}
\hline
Дата & Событие & Действие по исправлению & Примечание \\
\hline
\end{tabular}

\section{Замечания автора по существу работы}
Задание оказалось несложным, однако в качестве ознакомления с языком вызвало некие затруднения, которые были решены в процессе выполнения.

\section{Выводы}
В данной лабораторной работе я познакомился с языком Common Lisp и написал на нем программу, вычисляющую значение заданной функции. Программа работает за линейное время и память.

\end{document}