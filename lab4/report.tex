\documentclass[12pt]{article}

\usepackage{fullpage}
\usepackage{multicol,multirow}
\usepackage{tabularx}
\usepackage{ulem}
\usepackage[utf8]{inputenc}
\usepackage[russian]{babel}
\usepackage{amsmath}
\usepackage{amsmath}
\usepackage{minted}
\usepackage{titlesec}

\titleformat{\section}
  {\normalfont\Large\bfseries}{\thesection.}{0.3em}{}

\titleformat{\subsection}
  {\normalfont\large\bfseries}{\thesubsection.}{0.3em}{}

\titlespacing{\section}{0pt}{*2}{*2}
\titlespacing{\subsection}{0pt}{*1}{*1}
\titlespacing{\subsubsection}{0pt}{*0}{*0}
\usepackage{listings}
\lstloadlanguages{Lisp}
\lstset{extendedchars=false,
	breaklines=true,
	breakatwhitespace=true,
	keepspaces = true,
	tabsize=2
}
\begin{document}


\section*{Отчет по лабораторной работе №\,4 
по курсу \guillemotleft  Функциональное программирование\guillemotright}
\begin{flushright}
Студент группы 8О-308 МАИ \textit{Марков Александр}, \textnumero 15 по списку \\
\makebox[7cm]{Контакты: {\tt markov.lifeacc@gmail.com} \hfill} \\
\makebox[7cm]{Работа выполнена: 05.05.2021 \hfill} \\
\ \\
Преподаватель: Иванов Дмитрий Анатольевич, доц. каф. 806 \\
\makebox[7cm]{Отчет сдан: \hfill} \\
\makebox[7cm]{Итоговая оценка: \hfill} \\
\makebox[7cm]{Подпись преподавателя: \hfill} \\

\end{flushright}

\section{Тема работы}
Знаки и строки.

\section{Цель работы}
Научиться работать с литерами (знаками) и строками при помощи функций обработки строк и общих функций работы с последовательностями.

\section{Задание (вариант №4.9)}
Запрограммировать на языке Коммон Лисп функцию с двумя параметрами:

\begin{itemize}
    \item char - знак,
    \item sentence - строка предложения.
\end{itemize}

Функция должна подсчитать число вхождений знака char в последнее слово предложения sentence. Сравнение как латинских букв, так и русских должно быть регистро-независимым.

\section{Оборудование студента}
Процессор AMD Ryzen 5 4600H 3.00 GHz, память: 16Gb, разрядность системы: 64.

\section{Программное обеспечение}
ОС Windows 10, среда LispWorks Personal Edition 7.1.2

\section{Идея, метод, алгоритм}

Функция \textbf{last-word-char-count} принимает знак \textit{char} и предложение \textit{sentence}. Возвращает количество вхождений \textit{char} в последнее слово предложения \textit{sentence}. Функция работает следующим образом:

\begin{enumerate}
    \item Из предложения создается список слов с помощью функции из лекции \textbf{word-list};
    \item Берется последний элемент списка (последнее слово предложения);
    \item С помощью цикла рассматривается каждый символ последнего слова. Если рассматриваемый символ совпадает с \textit{char}, то увеличивается счётчик совпадений на единицу;
    \item Возвращается значение счётчика совпадений.
\end{enumerate}

Для корректного регистро-независимого сравнения как латинских символов, так и русских, используются функции из лекции: \textbf{russian-char-equal}, \textbf{russian-char-downcase}, \textbf{russian-upper-case-p}.

\section{Сценарий выполнения работы}

\section{Распечатка программы и её результаты}

\subsection{Исходный код}
\begin{minted}[
    frame=leftline,
    linenos
]{cl}
(defun word-list (sentence)
  (loop with len = (length sentence)
    for left = 0 then (1+ right)
    for right = (or (position-if #'whitespace-char-p sentence :start left)
                    len)
    unless (= right left)
      collect (subseq sentence left right)
    while (< right len)
  )
)

(defun russian-upper-case-p (char)
  (position char "АБВГДЕЁЖЗИЙКЛМНОПРСТУФХЦЧШЩЪЫЬЭЮЯ")
)

(defun russian-char-downcase (char)
  (let ((i (russian-upper-case-p char)))
    (if i
      (char "абвгдеёжзийклмнопрстуфхцчшщъыьэюя" i)
      (char-downcase char)   
    )
  )
)

(defun russian-char-equal (char1 char2)
  (char-equal (russian-char-downcase char1)
              (russian-char-downcase char2)
  )
)

(defun last-word-char-count (char sentence)
  (let* ((word-list (word-list sentence))
         (len-word-list (length word-list))
         (last-word (if (= (- len-word-list 1) -1)
                        NIL
                        (nth (- len-word-list 1) word-list)
           )
         )
         (count-matched 0)
        )
    (if (stringp last-word)
        (loop for ch across last-word
          do (when (russian-char-equal ch char)
                   (setf count-matched (+ 1 count-matched))
             )
        )
    )
    count-matched
  )
)
\end{minted}

\subsection{Результаты работы}

\begin{minted}[style=bw]{console}
CL-USER 8 > test1
"Я сразу смазал карту будня,"

CL-USER 9 > (last-word-char-count #\Д test1)
1

CL-USER 10 > (last-word-char-count #\д test1)
1

CL-USER 11 > (last-word-char-count #\z test1)
0

CL-USER 12 > test2
"плеснувши краску из стакана;"

CL-USER 13 > (last-word-char-count #\А test2)
3

CL-USER 14 > test3
"я показал на блюде студня"

CL-USER 15 > (last-word-char-count #\т test3)
1

CL-USER 16 > test4
"косые скулы океана."

CL-USER 17 > (last-word-char-count #\а test4)
2

CL-USER 18 > test5
""

CL-USER 19 > (last-word-char-count #\g test5)
0
\end{minted}

\section{Дневник отладки}
\begin{tabular}{|m{2cm}|m{6cm}|m{4cm}|m{3cm}|}
\hline
Дата & Событие & Действие по исправлению & Примечание \\
\hline
05.05.2021 & Происходила ошибка, если изначальное предложение было пустым & Был добавлен if при присваивании значения last-word, чтобы не было обращения к списку по несуществующему индексу & \\
\hline
\end{tabular}

\section{Замечания автора по существу работы}
Коммон Лисп предоставляет удобные средства для работы со строками.

\section{Выводы}
При выполнении данной лабораторной работы я научился работать со строками при помощи функций обработки строк и общих функций работы с последовательностями. Также при выполнении работы мне помогли знания, приобретенные в предыдущих работах.

\end{document}