\documentclass[12pt]{article}

\usepackage{fullpage}
\usepackage{multicol,multirow}
\usepackage{tabularx}
\usepackage{ulem}
\usepackage[utf8]{inputenc}
\usepackage[russian]{babel}
\usepackage{amsmath}
\usepackage{amsmath}

\usepackage{titlesec}

\titleformat{\section}
  {\normalfont\Large\bfseries}{\thesection.}{0.3em}{}

\titleformat{\subsection}
  {\normalfont\large\bfseries}{\thesubsection.}{0.3em}{}

\titlespacing{\section}{0pt}{*2}{*2}
\titlespacing{\subsection}{0pt}{*1}{*1}
\titlespacing{\subsubsection}{0pt}{*0}{*0}
\usepackage{listings}
\lstloadlanguages{Lisp}
\lstset{extendedchars=false,
	breaklines=true,
	breakatwhitespace=true,
	keepspaces = true,
	tabsize=2
}
\begin{document}


\section*{Отчет по лабораторной работе №\,3
по курсу \guillemotleft  Функциональное программирование\guillemotright}
\begin{flushright}
Студент группы 8О-308 МАИ \textit{Марков Александр}, \textnumero 15 по списку \\
\makebox[7cm]{Контакты: {\tt markov.lifeacc@gmail.com} \hfill} \\
\makebox[7cm]{Работа выполнена: 10.04.2021 \hfill} \\
\ \\
Преподаватель: Иванов Дмитрий Анатольевич, доц. каф. 806 \\
\makebox[7cm]{Отчет сдан: \hfill} \\
\makebox[7cm]{Итоговая оценка: \hfill} \\
\makebox[7cm]{Подпись преподавателя: \hfill} \\

\end{flushright}

\section{Тема работы}
Последовательности, массивы и управляющие конструкции Common Lisp.

\section{Цель работы}
Научиться создавать векторы и массивы для представления матриц, освоить общие функции работы с последовательностями, инструкции цикла и нелокального выхода.

\section{Задание (вариант №3.14)}
Запрограммировать на языке Коммон Лисп функцию, принимающую два аргумента:

\begin{itemize}
    \item A - двумерный массив, представляющий действительную матрицу произвольного размера,
    \item r - действительное число.
\end{itemize}

Функция должна найти наименьший элемент матрицы и вернуть новую матрицу того же размера, получающуюся из данной заменой всех вхождений наименьшего элемента на r.

Исходный массив должен оставаться неизменным.

\section{Оборудование студента}
Процессор AMD Ryzen 5 4600H 3.00 GHz, память: 16Gb, разрядность системы: 64.

\section{Программное обеспечение}
ОС Windows 10, среда LispWorks Personal Edition 7.1.2

\section{Идея, метод, алгоритм}

Задача была разбита на две подзадачи:

\begin{itemize}
    \item Поиск минимального элемента. Для этого была реализована функция \textbf{find-min};
    \item Создание нового двумерного массива и изменение всех входений наименьшего элемента. Для этого была реализована функция \textbf{change-arr}.
\end{itemize}

Функция \textbf{find-min} принимает двумерный массив, возвращает минимальный элемент и работает следующим образом:

\begin{itemize}
    \item С помощью вложенных циклов рассматривается каждый элемент двумерного массива.
    \item Если рассматриваемый элемент меньше, чем текущий наименьший, то обновляем значение наименьшего элемента.
\end{itemize}

Функий \textbf{change-arr} принимает двумерный массив $arr$ и действительное число $r$, возвращает новый двумерный массив, в котором все вхождения наименьшего элемента массива $arr$ заменены на $r$. Работает следующим образом:

\begin{itemize}
    \item Создается новый двумерный массив $new-arr$, имеющий размер такой же, что и у $arr$;
    \item С помощью функции \textbf{find-min} находится наименьший элемент массива $arr$;
    \item С помощью вложенных циклов рассматривается каждый элемент двумерного массива. Если этот элемент равен наименьшему, то на это же место в $new-arr$ ставится $r$. Иначе - элемент $arr$.
\end{itemize}

\section{Сценарий выполнения работы}

\section{Распечатка программы и её результаты}

\subsection{Исходный код}
\begin{lstlisting}
(defun print-matrix (matrix &optional (chars 3) stream)
  (let ((*print-right-margin* (+ 6 (* (+ 1 chars) (array-dimension matrix 1)))))
    (pprint matrix stream)
    (values)
  )
)

(defun find-min (arr)
  (let ((m (array-dimension arr 0))
        (n (array-dimension arr 1))
        (min-el (aref arr 0 0))
       )
    (loop for i upfrom 0 to (- m 1)
      do
      (loop for j upfrom 0 to (- n 1)
        do (when (< (aref arr i j) min-el)
             (setf min-el (aref arr i j))
           )
      )
    )
    min-el
  )
)

(defun change-arr (arr r)
  (let* ((m (array-dimension arr 0))
         (n (array-dimension arr 1))
         (new-arr (make-array (list m n) :initial-element 0.0))
         (min-el (find-min arr))
        )
    (loop for i upfrom 0 to (- m 1)
      do
      (loop for j upfrom 0 to (- n 1)
        do (when (= (aref arr i j) min-el)
             (setf (aref new-arr i j) r)
           )
           (when (/= (aref arr i j) min-el)
             (setf (aref new-arr i j) (aref arr i j))
           )
      )
    )
    new-arr
  )
)

\end{lstlisting}

\subsection{Результаты работы}

\begin{lstlisting}
CL-USER 17 > (print-matrix test1 6)

#2A((6.53 4321.4 431.44)
    (8.0 6.53 77.78)
    (123.12 88.132 6.53))

CL-USER 18 > (defvar result1 (change-arr test1 0.0))
RESULT1

CL-USER 19 > (print-matrix result1 6)

#2A((0.0 4321.4 431.44)
    (8.0 0.0 77.78)
    (123.12 88.132 0.0))

CL-USER 20 > (print-matrix test1 6)

#2A((6.53 4321.4 431.44)
    (8.0 6.53 77.78)
    (123.12 88.132 6.53))

CL-USER 21 > (print-matrix test2 6)

#2A((134.2 575.2 4321.4 55.55)
    (77.66 66.77 777.0 412.412)
    (431.44 12.33 33.12 1.0))

CL-USER 22 > (defvar result2 (change-arr test2 0.0))
RESULT2

CL-USER 23 > (print-matrix result2 7)

#2A((134.2 575.2 4321.4 55.55)
    (77.66 66.77 777.0 412.412)
    (431.44 12.33 33.12 0.0))

CL-USER 24 > (print-matrix test2 7)

#2A((134.2 575.2 4321.4 55.55)
    (77.66 66.77 777.0 412.412)
    (431.44 12.33 33.12 1.0))

CL-USER 25 > (print-matrix test3 4)

#2A((1.0 1.0 44.4)
    (5.0 21.2 41.4))

CL-USER 26 > (defvar result3 (change-arr test3 99.9))
RESULT3

CL-USER 27 > (print-matrix result3 4)

#2A((99.9 99.9 44.4)
    (5.0 21.2 41.4))

CL-USER 28 > (print-matrix test3 4)

#2A((1.0 1.0 44.4)
    (5.0 21.2 41.4))
\end{lstlisting}

\section{Дневник отладки}
\begin{tabular}{|m{2cm}|m{6cm}|m{4cm}|m{3cm}|}
\hline
Дата & Событие & Действие по исправлению & Примечание \\
\hline
& & & \\
\hline
\end{tabular}

\section{Замечания автора по существу работы}
Алгоритмически задача очень простая, но было достаточно интересно реализовать ее на Common Lisp.

\section{Выводы}
Массивы явлются основополагающей структурой данных в программировании и часто используются, поскольку в них удобно хранить данные. При выполнение данной лабораторной работы я познакомился с массивами в языке Common Lisp, а также попрактиковался с циклами.

\end{document}